\documentclass[11pt,a4paper]{moderncv}

% moderncv themes
%\moderncvtheme[blue]{casual}                 % optional argument are 'blue' (default), 'orange', 'red', 'green', 'grey' and 'roman' (for roman fonts, instead of sans serif fonts)
\moderncvtheme[blue]{classic}                % idem

\usepackage[T1]{fontenc}
% character encoding
\usepackage[utf8x]{inputenc}                   % replace by the encoding you are using
\usepackage[italian]{babel}

\renewcommand*{\namefont}{\fontsize{24}{24}\mdseries\upshape}

% adjust the page margins
\usepackage[scale=0.85]{geometry}
\recomputelengths                             % required when changes are made to page layout lengths

\fancyfoot{} % clear all footer fields
\fancyfoot[LE,RO]{\thepage}           % page number in "outer" position of footer line
\fancyfoot[RE,LO]{\footnotesize } % other info in "inner" position of footer line

% personal data
\firstname{M. en C. Becerril \smallskip Domínguez}  
\familyname{ \\ Óscar Azeem}
\title{Curriculum Vitae}               % optional, remove the line if not wanted
\address{}{México, CDMX}    % optional, remove the line if not wanted
\mobile{55 10789872}                    % optional, remove the line if not                % optional, remove the line if not wanted
%\phone{57994171}                      % optional, remove the line if not wanted
%\fax{312 996 1491}                          % optional, remove the line if not wanted
\email{admin@redirac.com}                      % optional, remove the line if not wanted
%\extrainfo{additional information (optional)} % optional, remove the line if not wanted
\photo[86pt]{imagen_curriculum_traje.png}          % '64pt' is the height the picture must be resized to and 'picture' is the name of the picture file; optional, remove the line if not wanted

%\quote{``El éxito es la habilidad de ir de fracaso en fracaso sin perder el entusiasmo.'' -- Winston Churchill}                 % optional, remove the line if not wanted

%----------------------------------------------------------------------------------
%            content
%----------------------------------------------------------------------------------
\begin{document}
\maketitle

\vspace{-10mm}

%Section
\section{Información}
%\cvcomputer{Nacimiento}{\small Octubre 22, 1992. en México (DF)}\normalsize}{Nacionalidad}{\small Mexicana \normalsize}

\cvcomputer{Nacimiento}{\small Octubre 22, 1992 (26 años). México.\normalsize}{\tiny .}{\small \tiny \normalsize}

%\cvcomputer{Additional Address}{J.Salvadora 14, Ciglenica, 44320 Kutina, Croatia}{Driving License}{\small B\normalsize}
\cvcomputer{LinkedIn}{\small \url{https://www.linkedin.com/in/oscarazeem/}\normalsize}{Webpage}{\small \url{http://redirac.com}\normalsize}
%\cvcomputer{Blog}{\small \url{http://ntrpivan.blogspot.com}\normalsize}{Skype}{\small ivan.ntrp}

\vspace{-2mm}

%Section
\section{Experiencia profesional}
\cvline{}{\Large Data Scientist y Desarrollador BI  (ETL)}
\cvline{}{\small Experiencia cómo desarrollador Business Intelligence (BI) en la compañía Bancomer, utilizando múltiples manejadores de bases de datos (DBMSs) tales cómo Oracle, Teradata y MySQL. Dichos DBMSs son utilizados para la Extracción, Transformación y Carga de tablas del tipo históricas y Staging, proceso conocido como ETL. Tal proceso ETL es realizado mediante el software BI Informatica PowerCenter y el lenguaje PL/SQL. La automatización es realizada mediante el software Control-M, junto con scripts BASH/KSH en un entorno UNIX/Linux. Experiencia cómo Analista de Datos en la compañía AT\&T, aplicando algoritmos de aprendizaje de máquina utilizando el lenguaje Python junto con las librerías ScikitLearn and Tensorflow con el propósito de clasificación de celdas 3G/4G por niveles de degradación; además de análisis mediante algoritmos de regresión para evaluar la probabilidad de fallo por cada celda. Experiencia cómo desarrollador Machine Learning mediante tesis de Maestría a través del desarrollo de un sistema de red de detección de intrusos (NIDS) utilizando algoritmos de aprendizaje de máquina tales cómo, KNN, Árboles de decisión, Bosques aleatorios, Máquina de soporte de vectores kernelizados y redes neuronales, todos utilizando los lenguajes Python/R, aplicados a la base de datos  Knowledge Discovery Database (KDD) y su versión más reciente NSLKDD. Todos los algoritmos de aprendizaje de máquina listados antes fueron aplicados cómo una función del tipo: fitness, dentro de algoritmos de búsqueda heurística tales como Random Mutation Hill Climbing (RMHC), Simulated Annealing (SA), Genetic Algorithms (GA) e Ions Motion Optimization Algorithm (IMO), con el objetivo de realizar una selección de características (dimensiones) relevantes en el dataset.}


\vspace{-4mm}

%Section
\section{Habilidades computacionales y metodologías de desarrollo Agile} 
\cvline{Lenguajes}{Python, R, SQL, PL/SQL, BASH, KSH, \small C, C++, Matlab, ASM, \LaTeX}
\cvline{DBMSs}{Teradata, Oracle, MySQL}
\cvline{BI}{Informatica PowerCenter, Workflow, Designer, Repository, Monitor}
\cvline{Plataformas}{Unix, Linux, Windows, Control-M, MAC OSX}  {}{}
\cvline{Tools - Librerías}{
	Git, Github, ScikitLearn, Tensorflow, Numpy, Pandas, Matplotlib, Seaborn, Teradata Studio, Oracle Developer, MySQL Workbench,
	Emacs, Eclipse IDE, Sublime Text, TexStudio, Office Suite (Excel, Word, Power Point), Jupyter Notebook.}
\cvline{Agile}{Scrum}  {}{}

\vspace{-3mm}


\newpage


%Section
\section{Experiencia Laboral}

\cventry{06/2018-actualmente}{Bancomer - Softtek}{Business Intelligence}{Desarrollador ETL - Informatica PowerCenter. Oracle DBA}{}{}{}
\vspace{-1mm}

\cventry{03/2018-06/2018}{AT\&T - Ibérica}{Analista de Datos Jr}{Área Triage. Monitoreo de la calidad de las celdas 3G y 4G.}{}{}
\vspace{-1mm}


%Section

\section{Educación}
\cventry{2015-2018}{Maestría en Ciencias en Ingeniería de Telecomunicaciones}{\href{http://www.sepi.esimez.ipn.mx//}{Instituto Politécnico Nacional}}{SEPI-ESIME}{Zacatenco}{Posgrado. Especialidad en redes de cómputo y machine learning. Créditos completados.}

\cventry{2010-2015}{Ingeniería en comunicaciones y electrónica}{\href{http://www.esimez.ipn.mx//}{Instituto Politécnico Nacional}}{ESIME}{Zacatenco}{Grado Universitario. Especialidad en computación con cédula profesional.}
  
%\cventry{2007-2010}{Técnico en Redes de cómputo}{\href{http://www.cet1.ipn.mx/}{Instituto Politécnico Nacional, CET 1}}{Walter Cross Buchanan}{Grado de Bachillerato}{}

%\cventry{2004-2007}{Técnico en Secretariado}{Escuela Secundaria Técnica}{No. 90}{Arturo Rosenblueth Stearns}{Educación Secundaria}{}
\vspace{-3mm}

%\newpage
\section{Lenguajes Adicionales}
%\hspace{25mm}\small  Porcentajes de Idiomas \href{}{}
%\vspace{5mm}
Veinte niveles cursados en la escuela de inglés: Quick Learning. Nivel B2 avalado por el Centro de Lenguas Extranjeras (CENLEX) del Instituto Politécnico Nacional 

\begin{tabular}{p{67mm} p{40mm} p{45mm} p{20mm}}
& \textbf{Comprensión} & \textbf{Hablado} & \textbf{Escrito} \\
\end{tabular}

\begin{tabular}{p{67mm} p{27mm} p{13mm} p{20mm} p{20mm} p{20mm}}
& Escuchado &  Leído & Interacción & Generación & \\
\end{tabular}

\vspace{3mm}
%\cvlanguage{Español}{Lengua Materna}{
%	\begin{tabular}{ p{20mm} p{20mm} p{20mm} p{20mm} p{21mm}}
%		90\% & 90\% & 90\% & 90\% & 90\%
%	\end{tabular}}
\cvlanguage{Inglés}{Avanzado}{
	\begin{tabular}{ p{20mm} p{20mm} p{20mm} p{20mm} p{21mm}}
		95\% & 100\% & 90\% & 90\% & 95\%
	\end{tabular}}

\section{Cursos}
\cventry{2018-Softtek}{ETL - Informatica PowerCenter, SCRUM, SQL}{}{}{}{}
\cventry{" -Coursera}{IBM - Data Scientist}{}{}{}{}
%\cventry{2017-Udemy}{Python A-Z: Python For Data Science, Bash Programming Course, The Complete Wireshark Course, CCNA 2017 200-125 Video Boot Camp}{}{}{}{}
\vspace{-1mm}


\section{Conferencias}

\cventry{2018}{Congreso Nacional de Expo Acustica}{http://www.expoacustica.ipn.mx/}{Tema: ``Reconocimiento de voz utilizando redes neuronales''}{México. CDMX}{}{}{}

\cventry{2017}{XVI Congreso Nacional de Ingeniería Electromecánica y de sistemas}{http://www.sepi.esimez.ipn.mx/cnies/}{Tema: ``Algoritmo IMO y SVMs para la elaboración de un sistema de red de detección de intrusos ligero basado en anomalías mediante la base de datos NSL-KDD}{México. CDMX}{}{}{}

\cventry{2016}{Coloquio de Seguridad en Redes de Computadoras}{Tema: ``Sistemas de red de detección de intrusos y Machine learning''}{México. CDMX}{}{}{}

\ %cventry{2016}{Congreso Nacional de Expo Acustica}{http://www.expoacustica.ipn.mx/}{Topic: ``Uso de algoritmos de ML en el reconocimiento de voz''}{México. CDMX}{}{}{}


\vspace{-3mm}


\end{document}