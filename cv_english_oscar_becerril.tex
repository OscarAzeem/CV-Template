\documentclass[11pt,a4paper]{moderncv}

% moderncv themes
%\moderncvtheme[blue]{casual}                 % optional argument are 'blue' (default), 'orange', 'red', 'green', 'grey' and 'roman' (for roman fonts, instead of sans serif fonts)
\moderncvtheme[blue]{classic}                % idem

\usepackage[T1]{fontenc}
% character encoding
\usepackage[utf8x]{inputenc}                   % replace by the encoding you are using
\usepackage[italian]{babel}

\renewcommand*{\namefont}{\fontsize{24}{32}\mdseries\upshape}


% adjust the page margins
\usepackage[scale=0.85]{geometry}
\recomputelengths                             % required when changes are made to page layout lengths

\fancyfoot{} % clear all footer fields
\fancyfoot[LE,RO]{\thepage}           % page number in "outer" position of footer line
\fancyfoot[RE,LO]{\footnotesize } % other info in "inner" position of footer line

% personal data
\firstname{M.S. Becerril Domínguez} %\smallskip  
\familyname{ \\Óscar Azeem}
\title{Curriculum Vitae}               % optional, remove the line if not wanted
\address{México, CDMX}{}    % optional, remove the line if not wanted
\mobile{55 10789872}                    % optional, remove the line if not wanted
%\phone{}                      % optional, remove the line if not wanted
%\fax{312 996 1491}                          % optional, remove the line if not wanted
\email{admin@redirac.com}                      % optional, remove the line if not wanted
%\extrainfo{additional information (optional)} % optional, remove the line if not wanted
\photo[86pt]{imagen_curriculum_traje.png}          % '64pt' is the height the picture must be resized to and 'picture' is the name of the picture file; optional, remove the line if not wanted

%\quote{``El éxito es la habilidad de ir de fracaso en fracaso sin perder el entusiasmo.'' -- Winston Churchill}                 % optional, remove the line if not wanted

%----------------------------------------------------------------------------------
%            content
%----------------------------------------------------------------------------------
\begin{document}
\maketitle

\vspace{-10mm}

%Section
\section{Information}
%\cvcomputer{Nacimiento}{\small Octubre 22, 1992. en México (DF)}\normalsize}{Nacionalidad}{\small Mexicana \normalsize}

\cvcomputer{Born}{\small October 22, 1992 (26 years old). Mexico.\normalsize}{\tiny .}{\small \tiny \normalsize}

%\cvcomputer{Additional Address}{J.Salvadora 14, Ciglenica, 44320 Kutina, Croatia}{Driving License}{\small B\normalsize}
\cvcomputer{LinkedIn}{\small \url{https://www.linkedin.com/in/oscarazeem/}\normalsize}{Webpage}{\small \url{http://redirac.com}\normalsize}
%\cvcomputer{Blog}{\small \url{http://ntrpivan.blogspot.com}\normalsize}{Skype}{\small ivan.ntrp}

\vspace{-2mm}

%Section
\section{Professional Experience}
\cvline{}{%Header
	\Large Data Scientist and BI developer}
\cvline{}{%Descripción
	\small
	%start
	%
Experience as a Business Intelligence (BI) developer at the Bancomer company,  using multiple Databases Management Systems (DBMSs) such as Oracle, Teradata, MySQL; for extraction, transformation, and load of historical/staging tables; processes known as ETL.  Such process is done through the Informatica PowerCenter BI software and the PL/SQL language. Automation it's done by BASH/KSH scripts in a Unix/Linux environment. 
Experience as Data Analyst at the AT\&T company, applying machine learning algorithms through Python using the ScikitLearn and Tensorflow libraries, with the purpose of 3G/4G cells classification by degradation levels; also regression analysis for failure probability for each cell. 
Experience as Python/R Machine Learning developer by Master's Thesis by the development of a Network Intrusion Detection System (NIDS) using Machine Learning Algorithms as KNN, Decision Threes, Random Forest, Kernelized Support Vector Machines, Neuronal Networks, all using Python/R, applied to the Knowledge Discovery Database (KDD) with four million of vectors and NSLKDD (a newest version).  All the algorithms listed before were applied as a fitness function within the heuristic search algorithms such as Random Mutation Hill Climbing (RMHC), Simulated Annealing (SA), Genetic Algorithms (GA) e Ions Motion Optimization Algorithm (IMO), all with the purpose of selecting relevant features  (feature selection) within the 41 columns dataset.
	%
%end
}


\vspace{-4mm}

%Section
\section{Computational Skills and Agile Software Development Methodologies } 
\cvline{Languages}{Python, R, SQL, PL/SQL, BASH, KSH, \small C, C++, Matlab, ASM, \LaTeX}
\cvline{DBMSs}{Teradata, Oracle, MySQL}
\cvline{BI}{Informatica PowerCenter, Workflow, Designer, Repository, Monitor}
\cvline{Platforms}{Unix, Linux, Windows, Control-M, MAC OSX}  {}{}
\cvline{Tools - Libraries}{
	Git, Github, ScikitLearn, Tensorflow, Numpy, Pandas, Matplotlib, Seaborn, Teradata Studio, Oracle Developer, MySQL Workbench,
	Emacs, Eclipse IDE, Sublime Text, TexStudio, Office Suite (Excel, Word, Power Point), Jupyter Notebook.}
\cvline{Agile}{Scrum}  {}{}

\vspace{-3mm}

\newpage

%Section
\section{Work Experience}


\cventry{06/2018-current}{Bancomer - Softtek}{Business Intelligence - ETL (IPC) Developer}{\newline \textbf{Activities:} Use of the pmrep and pmcmd commands from Informatica PowerCenter inside BASH/KSH scripts to automate workflow execution, extraction, importation and exportation. Load of fixed with and delimited flat files such as CSV, TSV, COBOL to multiple staging or historical tables applying IPC transformations like Expression, Filter, LookUp, UpdateStrategy, Joiner, Union, SQL, Stored procedure, Normalizer. PL/SQL query translation to IPC mapping/workflows complying the Pushdown optimization.}{}{}
\vspace{-1mm}

\cventry{04/2018-06/2018}{AT\&T - Ibérica}{Jr. Data Analyst}{\newline \textbf{Activities:} Develop of machine learning models such as Random Forest, Polynomial regression, Support Vector Machines, for 3G and 4G cells classification and failure probability using the Python language. Load of CSV files using the Python Pandas library to filter, mining and export information to an Oracle server for further analysis. }{}{}
\vspace{-1mm}

%\cventry{2016/01-2015/10}{Link Consulting}{RF Ingeniero de Post Proceso}{\newline \textbf{Actividades:} Análisis de datos para la realización de Post proceso Indoor y OUtdoor, para red: WCDMA y GSM}{\newline \textbf{Herramientas:} Genex Assistant v 14 y Genex Probe v.14}{}


\section{Education}
\cventry{2016-2018}{M.S. Telecommunications Engineering}{\href{http://www.sepi.esimez.ipn.mx//}{Instituto Politécnico Nacional}}{Sección de Estudios de Posgrado e Invesgigación (SEPI) ESIME}{Zacatenco}{Master's Degree. Network computer and Machine Learning speciality.}

\cventry{2011-2016}{Communications and Electronics Engineering}{\href{http://www.esimez.ipn.mx//}{Instituto Politécnico Nacional}}{ESIME}{Zacatenco}{Bachelor's Degree. Computation speciality with professional license}

%Section
\section{Languages}
%\hspace{25mm}\small  Porcentajes de Idiomas \href{}{}
%\vspace{5mm}
Twenty levels completed at the Quick Learning English school. Level B2 endorsed by the Centro de Lenguas Extranjeras del Instituto Politécnico Nacional (CENLEX) English school.
\\

\begin{tabular}{p{67mm} p{40mm} p{45mm} p{20mm}}
& \textbf{Comprehension} & \textbf{Speaking} & \textbf{Writing} \\
\end{tabular}


\begin{tabular}{p{67mm} p{20mm} p{20mm} p{20mm} p{20mm} p{20mm}}
& Listening & Reading   & Interaction & Production & \\
\end{tabular}

\vspace{3mm}
\cvlanguage{Spanish}{Mother tongue}{
	\begin{tabular}{ p{20mm} p{20mm} p{20mm} p{20mm} p{21mm}}
		100\% & 100\% & 100\% & 100\% & 100\%
	\end{tabular}}
\cvlanguage{English}{Advanced}{
	\begin{tabular}{ p{20mm} p{20mm} p{20mm} p{20mm} p{21mm}}
		95\% & 100\% & 90\% & 90\% & 95\%
	\end{tabular}}


%Section





\section{Courses}
\cventry{2018-Softtek}{ETL - Informatica PowerCenter, SCRUM, SQL}{}{}{}{}
\cventry{" -Coursera}{IBM - Data Scientist}{}{}{}{}
%\cventry{2017-Udemy}{Python A-Z: Python For Data Science, Bash Programming Course, The Complete Wireshark Course, CCNA 2017 200-125 Video Boot Camp}{}{}{}{}
\vspace{-1mm}


\section{Conferences}

\cventry{2018}{Congreso Nacional de Expo Acustica}{http://www.expoacustica.ipn.mx/}{Topic: ``Reconocimiento de voz utilizando redes neuronales''}{México. CDMX}{}{}{}

\cventry{2017}{XVI Congreso Nacional de Ingeniería Electromecánica y de sistemas}{http://www.sepi.esimez.ipn.mx/cnies/}{Topic: ``Algoritmo IMO y SVMs para la elaboración de un sistema de red de detección de intrusos ligero basado en anomalías mediante la base de datos NSL-KDD}{México. CDMX}{}{}{}

\cventry{2016}{Coloquio de Seguridad en Redes de Computadoras}{Topic: ``Sistemas de red de detección de intrusos y Machine learning''}{México. CDMX}{}{}{}

\ %cventry{2016}{Congreso Nacional de Expo Acustica}{http://www.expoacustica.ipn.mx/}{Topic: ``Uso de algoritmos de ML en el reconocimiento de voz''}{México. CDMX}{}{}{}


\vspace{-3mm}


\end{document}